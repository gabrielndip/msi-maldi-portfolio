% Options for packages loaded elsewhere
\PassOptionsToPackage{unicode}{hyperref}
\PassOptionsToPackage{hyphens}{url}
\PassOptionsToPackage{dvipsnames,svgnames,x11names}{xcolor}
%
\documentclass[
  letterpaper,
  DIV=11,
  numbers=noendperiod]{scrartcl}

\usepackage{amsmath,amssymb}
\usepackage{iftex}
\ifPDFTeX
  \usepackage[T1]{fontenc}
  \usepackage[utf8]{inputenc}
  \usepackage{textcomp} % provide euro and other symbols
\else % if luatex or xetex
  \usepackage{unicode-math}
  \defaultfontfeatures{Scale=MatchLowercase}
  \defaultfontfeatures[\rmfamily]{Ligatures=TeX,Scale=1}
\fi
\usepackage{lmodern}
\ifPDFTeX\else  
    % xetex/luatex font selection
\fi
% Use upquote if available, for straight quotes in verbatim environments
\IfFileExists{upquote.sty}{\usepackage{upquote}}{}
\IfFileExists{microtype.sty}{% use microtype if available
  \usepackage[]{microtype}
  \UseMicrotypeSet[protrusion]{basicmath} % disable protrusion for tt fonts
}{}
\makeatletter
\@ifundefined{KOMAClassName}{% if non-KOMA class
  \IfFileExists{parskip.sty}{%
    \usepackage{parskip}
  }{% else
    \setlength{\parindent}{0pt}
    \setlength{\parskip}{6pt plus 2pt minus 1pt}}
}{% if KOMA class
  \KOMAoptions{parskip=half}}
\makeatother
\usepackage{xcolor}
\setlength{\emergencystretch}{3em} % prevent overfull lines
\setcounter{secnumdepth}{-\maxdimen} % remove section numbering
% Make \paragraph and \subparagraph free-standing
\makeatletter
\ifx\paragraph\undefined\else
  \let\oldparagraph\paragraph
  \renewcommand{\paragraph}{
    \@ifstar
      \xxxParagraphStar
      \xxxParagraphNoStar
  }
  \newcommand{\xxxParagraphStar}[1]{\oldparagraph*{#1}\mbox{}}
  \newcommand{\xxxParagraphNoStar}[1]{\oldparagraph{#1}\mbox{}}
\fi
\ifx\subparagraph\undefined\else
  \let\oldsubparagraph\subparagraph
  \renewcommand{\subparagraph}{
    \@ifstar
      \xxxSubParagraphStar
      \xxxSubParagraphNoStar
  }
  \newcommand{\xxxSubParagraphStar}[1]{\oldsubparagraph*{#1}\mbox{}}
  \newcommand{\xxxSubParagraphNoStar}[1]{\oldsubparagraph{#1}\mbox{}}
\fi
\makeatother

\usepackage{color}
\usepackage{fancyvrb}
\newcommand{\VerbBar}{|}
\newcommand{\VERB}{\Verb[commandchars=\\\{\}]}
\DefineVerbatimEnvironment{Highlighting}{Verbatim}{commandchars=\\\{\}}
% Add ',fontsize=\small' for more characters per line
\usepackage{framed}
\definecolor{shadecolor}{RGB}{241,243,245}
\newenvironment{Shaded}{\begin{snugshade}}{\end{snugshade}}
\newcommand{\AlertTok}[1]{\textcolor[rgb]{0.68,0.00,0.00}{#1}}
\newcommand{\AnnotationTok}[1]{\textcolor[rgb]{0.37,0.37,0.37}{#1}}
\newcommand{\AttributeTok}[1]{\textcolor[rgb]{0.40,0.45,0.13}{#1}}
\newcommand{\BaseNTok}[1]{\textcolor[rgb]{0.68,0.00,0.00}{#1}}
\newcommand{\BuiltInTok}[1]{\textcolor[rgb]{0.00,0.23,0.31}{#1}}
\newcommand{\CharTok}[1]{\textcolor[rgb]{0.13,0.47,0.30}{#1}}
\newcommand{\CommentTok}[1]{\textcolor[rgb]{0.37,0.37,0.37}{#1}}
\newcommand{\CommentVarTok}[1]{\textcolor[rgb]{0.37,0.37,0.37}{\textit{#1}}}
\newcommand{\ConstantTok}[1]{\textcolor[rgb]{0.56,0.35,0.01}{#1}}
\newcommand{\ControlFlowTok}[1]{\textcolor[rgb]{0.00,0.23,0.31}{\textbf{#1}}}
\newcommand{\DataTypeTok}[1]{\textcolor[rgb]{0.68,0.00,0.00}{#1}}
\newcommand{\DecValTok}[1]{\textcolor[rgb]{0.68,0.00,0.00}{#1}}
\newcommand{\DocumentationTok}[1]{\textcolor[rgb]{0.37,0.37,0.37}{\textit{#1}}}
\newcommand{\ErrorTok}[1]{\textcolor[rgb]{0.68,0.00,0.00}{#1}}
\newcommand{\ExtensionTok}[1]{\textcolor[rgb]{0.00,0.23,0.31}{#1}}
\newcommand{\FloatTok}[1]{\textcolor[rgb]{0.68,0.00,0.00}{#1}}
\newcommand{\FunctionTok}[1]{\textcolor[rgb]{0.28,0.35,0.67}{#1}}
\newcommand{\ImportTok}[1]{\textcolor[rgb]{0.00,0.46,0.62}{#1}}
\newcommand{\InformationTok}[1]{\textcolor[rgb]{0.37,0.37,0.37}{#1}}
\newcommand{\KeywordTok}[1]{\textcolor[rgb]{0.00,0.23,0.31}{\textbf{#1}}}
\newcommand{\NormalTok}[1]{\textcolor[rgb]{0.00,0.23,0.31}{#1}}
\newcommand{\OperatorTok}[1]{\textcolor[rgb]{0.37,0.37,0.37}{#1}}
\newcommand{\OtherTok}[1]{\textcolor[rgb]{0.00,0.23,0.31}{#1}}
\newcommand{\PreprocessorTok}[1]{\textcolor[rgb]{0.68,0.00,0.00}{#1}}
\newcommand{\RegionMarkerTok}[1]{\textcolor[rgb]{0.00,0.23,0.31}{#1}}
\newcommand{\SpecialCharTok}[1]{\textcolor[rgb]{0.37,0.37,0.37}{#1}}
\newcommand{\SpecialStringTok}[1]{\textcolor[rgb]{0.13,0.47,0.30}{#1}}
\newcommand{\StringTok}[1]{\textcolor[rgb]{0.13,0.47,0.30}{#1}}
\newcommand{\VariableTok}[1]{\textcolor[rgb]{0.07,0.07,0.07}{#1}}
\newcommand{\VerbatimStringTok}[1]{\textcolor[rgb]{0.13,0.47,0.30}{#1}}
\newcommand{\WarningTok}[1]{\textcolor[rgb]{0.37,0.37,0.37}{\textit{#1}}}

\providecommand{\tightlist}{%
  \setlength{\itemsep}{0pt}\setlength{\parskip}{0pt}}\usepackage{longtable,booktabs,array}
\usepackage{calc} % for calculating minipage widths
% Correct order of tables after \paragraph or \subparagraph
\usepackage{etoolbox}
\makeatletter
\patchcmd\longtable{\par}{\if@noskipsec\mbox{}\fi\par}{}{}
\makeatother
% Allow footnotes in longtable head/foot
\IfFileExists{footnotehyper.sty}{\usepackage{footnotehyper}}{\usepackage{footnote}}
\makesavenoteenv{longtable}
\usepackage{graphicx}
\makeatletter
\def\maxwidth{\ifdim\Gin@nat@width>\linewidth\linewidth\else\Gin@nat@width\fi}
\def\maxheight{\ifdim\Gin@nat@height>\textheight\textheight\else\Gin@nat@height\fi}
\makeatother
% Scale images if necessary, so that they will not overflow the page
% margins by default, and it is still possible to overwrite the defaults
% using explicit options in \includegraphics[width, height, ...]{}
\setkeys{Gin}{width=\maxwidth,height=\maxheight,keepaspectratio}
% Set default figure placement to htbp
\makeatletter
\def\fps@figure{htbp}
\makeatother

\KOMAoption{captions}{tableheading}
\makeatletter
\@ifpackageloaded{caption}{}{\usepackage{caption}}
\AtBeginDocument{%
\ifdefined\contentsname
  \renewcommand*\contentsname{Table of contents}
\else
  \newcommand\contentsname{Table of contents}
\fi
\ifdefined\listfigurename
  \renewcommand*\listfigurename{List of Figures}
\else
  \newcommand\listfigurename{List of Figures}
\fi
\ifdefined\listtablename
  \renewcommand*\listtablename{List of Tables}
\else
  \newcommand\listtablename{List of Tables}
\fi
\ifdefined\figurename
  \renewcommand*\figurename{Figure}
\else
  \newcommand\figurename{Figure}
\fi
\ifdefined\tablename
  \renewcommand*\tablename{Table}
\else
  \newcommand\tablename{Table}
\fi
}
\@ifpackageloaded{float}{}{\usepackage{float}}
\floatstyle{ruled}
\@ifundefined{c@chapter}{\newfloat{codelisting}{h}{lop}}{\newfloat{codelisting}{h}{lop}[chapter]}
\floatname{codelisting}{Listing}
\newcommand*\listoflistings{\listof{codelisting}{List of Listings}}
\makeatother
\makeatletter
\makeatother
\makeatletter
\@ifpackageloaded{caption}{}{\usepackage{caption}}
\@ifpackageloaded{subcaption}{}{\usepackage{subcaption}}
\makeatother

\ifLuaTeX
  \usepackage{selnolig}  % disable illegal ligatures
\fi
\usepackage{bookmark}

\IfFileExists{xurl.sty}{\usepackage{xurl}}{} % add URL line breaks if available
\urlstyle{same} % disable monospaced font for URLs
\hypersetup{
  pdftitle={MSI MALDI Portfolio Workflow},
  colorlinks=true,
  linkcolor={blue},
  filecolor={Maroon},
  citecolor={Blue},
  urlcolor={Blue},
  pdfcreator={LaTeX via pandoc}}


\title{MSI MALDI Portfolio Workflow}
\author{}
\date{2025-11-13}

\begin{document}
\maketitle

\renewcommand*\contentsname{Table of contents}
{
\hypersetup{linkcolor=}
\setcounter{tocdepth}{3}
\tableofcontents
}

\section{Introduction}\label{introduction}

This document outlines the complete workflow for processing and
analyzing a MALDI mass‑spectrometry imaging (MSI) dataset.

The overall goal is to load raw MSI data, perform a series of
preprocessing steps (normalisation, peak picking, alignment and
filtering), inspect and visualise the data, optionally crop a region of
interest (ROI), carry out spatial segmentation and principal component
analysis, and finish with optional downstream analyses. Each major step
is encapsulated in a separate script and sourced in this Quarto document
for clarity and reproducibility.

\subsection{Data and project
structure}\label{data-and-project-structure}

The data directory contains the following files:

\begin{verbatim}
data
├── MB001-040_pos.tif          # Optical scan of the tissue section
├── N-glycan_RScript.docx       # Original analysis notes
├── README.txt                  # Dataset description
├── cal-nonormalization.ibd     # Binary data for the calibrant region
├── cal-nonormalization.imzML   # Metadata for the calibrant region
├── control-nonormalization.ibd # Binary data for the control region
├── control-nonormalization.imzML
├── png1-nonormalization.ibd    # Binary data for treated region 1
├── png1-nonormalization.imzML
├── png2-nonormalization.ibd    # Binary data for treated region 2
└── png2-nonormalization.imzML
\end{verbatim}

\subsection{Workflow overview}\label{workflow-overview}

Each of the following sections sources an R script from the
\texttt{scripts/} directory. The scripts are created separately and
contain the code required to execute each step. The Quarto document
executes the scripts sequentially when rendered. If you modify script
file names, update the paths accordingly.

\section{Data loading}\label{data-loading}

Load the raw imzML and ibd files into a Cardinal \texttt{MSImageSet}
object.

\begin{Shaded}
\begin{Highlighting}[]
\FunctionTok{source}\NormalTok{(}\StringTok{"scripts/01\_load\_data.R"}\NormalTok{)}
\end{Highlighting}
\end{Shaded}

\section{Preprocessing}\label{preprocessing}

Perform total‑ion current (TIC) normalization, peak picking, alignment
and filtering. Adjust thresholds and tolerances within the script as
required.

\begin{Shaded}
\begin{Highlighting}[]
\FunctionTok{source}\NormalTok{(}\StringTok{"scripts/02\_preprocess\_data.R"}\NormalTok{)}
\end{Highlighting}
\end{Shaded}

\section{Quality control and
exploration}\label{quality-control-and-exploration}

Generate diagnostic plots, ion images and spectrum views to assess data
quality both before and after preprocessing.

\begin{Shaded}
\begin{Highlighting}[]
\FunctionTok{source}\NormalTok{(}\StringTok{"scripts/03\_explore\_data.R"}\NormalTok{)}
\end{Highlighting}
\end{Shaded}

\section{ROI selection and cropping}\label{roi-selection-and-cropping}

Optionally crop the dataset to a region of interest. This step is useful
for speeding up exploration or focusing on a particular anatomical area.
If you prefer to analyse the full dataset, you may omit this section.

\begin{Shaded}
\begin{Highlighting}[]
\FunctionTok{source}\NormalTok{(}\StringTok{"scripts/04\_roi\_crop.R"}\NormalTok{)}
\end{Highlighting}
\end{Shaded}

\section{Segmentation: Identifying Molecular
Regions}\label{segmentation-identifying-molecular-regions}

This step partitions the image into distinct molecular regions based on
spectral similarity. We will apply and compare three different
clustering methods to understand the benefits of spatially-aware
algorithms.

The script \texttt{05\_ssc\_segmentation.R} now performs two main tasks:
1. \textbf{Optimizing \texttt{k}}: It first runs the Spatial Shrunken
Centroids (SSC) algorithm for a range of \texttt{k} values (number of
segments), plotting each result to help us choose a \texttt{k} that best
fits the data's complexity. 2. \textbf{Comparing Methods}: For a fixed
\texttt{k}, it generates a side-by-side comparison of three methods: -
\textbf{Spatial Shrunken Centroids (SSC)}: A sophisticated method that
incorporates spatial smoothing and performs feature selection. -
\textbf{Spatial K-Means (SKM)}: A faster spatial method that adds a
smoothing penalty to the standard k-means algorithm. - \textbf{K-Means
(non-spatial)}: Standard k-means clustering performed on the PCA scores,
used here as a baseline.

\begin{Shaded}
\begin{Highlighting}[]
\FunctionTok{source}\NormalTok{(}\StringTok{"scripts/05\_ssc\_segmentation.R"}\NormalTok{)}
\end{Highlighting}
\end{Shaded}

\subsubsection{Interpreting the Results}\label{interpreting-the-results}

The code chunk above will produce two sets of plots:

\begin{enumerate}
\def\labelenumi{\arabic{enumi}.}
\item
  \textbf{SSC Results (Varying k)}: This plot shows the tissue
  segmentation for each \texttt{k} value tested. By visually inspecting
  these images, you can choose a \texttt{k} that appears to best capture
  the underlying anatomical structures without over-segmenting the
  tissue into noisy, meaningless regions.
\item
  \textbf{Segmentation Method Comparison}: This 3-panel plot directly
  contrasts the output of SSC, SKM, and non-spatial K-Means. You will
  likely observe that the K-Means result has a noisy,
  ``salt-and-pepper'' appearance. In contrast, the SKM and SSC results
  should appear much smoother and more contiguous, as these algorithms
  penalize solutions where neighboring
\end{enumerate}

\begin{itemize}
\tightlist
\item
  pixels are assigned to different clusters. This demonstrates the power
  of including spatial information in the clustering process.
\end{itemize}

\section{Principal component analysis
(PCA)}\label{principal-component-analysis-pca}

Perform dimensionality reduction (e.g.~PCA) to extract dominant spectral
patterns and visualize spatial loadings.

\begin{Shaded}
\begin{Highlighting}[]
\FunctionTok{source}\NormalTok{(}\StringTok{"scripts/06\_pca\_analysis.R"}\NormalTok{)}
\end{Highlighting}
\end{Shaded}

\section{Additional analyses}\label{additional-analyses}

Include any extra analyses not covered above, such as t‑SNE/UMAP,
supervised classification, co‑localisation networks or annotation-based
exploration. Create additional scripts as required and source them here.

\begin{Shaded}
\begin{Highlighting}[]
\FunctionTok{source}\NormalTok{(}\StringTok{"scripts/07\_extra\_analysis.R"}\NormalTok{)}
\end{Highlighting}
\end{Shaded}

\section{Conclusions}\label{conclusions}

Summarise your findings from each analysis step and discuss any
biological interpretations or follow‑up questions. This section can also
document challenges encountered and potential improvements for future
analyses.




\end{document}
